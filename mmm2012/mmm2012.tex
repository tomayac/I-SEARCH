\documentclass[runningheads,a4paper]{llncs}
\usepackage[utf8]{inputenc}
\usepackage[hyphens]{url}
\usepackage{graphicx}
\usepackage{hyperref}
\usepackage{float}
\usepackage{eurosym}
\usepackage[normalem]{ulem}
\usepackage{alltt}
\usepackage{subfig}
\usepackage{amssymb}
\setcounter{tocdepth}{3}

\usepackage{url}
\newcommand{\keywords}[1]{\par\addvspace\baselineskip
\noindent\keywordname\enspace\ignorespaces#1}

\begin{document}

\mainmatter  % start of an individual contribution

% first the title is needed
\title{Context-aware Querying for\\ Multi-modal Search Engines}

% a short form should be given in case it is too long for the running head
\titlerunning{Context-aware Querying for Multi-modal Search Engines}

\author{Jonas Etzold\inst{1}, Arnaud Brousseau\inst{2} Paul Grimm\inst{1} and Thomas Steiner\inst{2}}

\authorrunning{Context-aware Querying for Multi-modal Search Engines}
% (feature abused for this document to repeat the title also on left hand pages)

% the affiliations are given next; don't give your e-mail address
% unless you accept that it will be published
\institute{Erfurt University of Applied Sciences, Germany, \email{\{jonas.etzold|grimm\}@fh-erfurt.de}
\and Google Germany GmbH, ABC-Str. 19, 20354 Hamburg, Germany, \email{\{arnaudb|tomac\}@google.com}}

\maketitle

\begin{abstract}
(Tom)
\keywords{lorem, ipsum}
\end{abstract}

\section{Introduction}
(Tom) \cite{ijmis}

\section{Related Work}
(Jonas) \cite{nigay}

\section{Methodology}
(Arnaud, Jonas)

\subsection{MMBag}

\begin{itemize}
\item A common pattern in search accross the whole web: a single input box (Yahoo, Bing, Google). Users are used to search with this pattern. More importantly, they \emph{expect} a search engine to use this pattern. I-SEARCH will be different from the search engine we've seen so far since it accepts a large number of types of input: audio, video, 3D objects, sketches, etc. How can we do this?
 
\item Quick look at TinEye, MMRetrieval and Search-by-image (Google). There's obviously a big UX problem. Google seems to get it right since it keeps the text field paradigm, but I-SEARCH can't afford to put 7/8 different icons in a text box.
\item 2 possible solutions to solve the problem: forget the text-box pattern, or adapt it. We chose to adapt it: a text-box, enriched with "tokenization" possibility and autocompletion. We decided to make a separated menu, while keeping the main text-field reactive to events.
\end{itemize}

\subsection{UIIFace}

\subsection{CoFind}

\section{Implementation and Results}
(Jonas, Arnaud)
Talking about the interface itself: 
\begin{itemize}
\item HTML5
\item CSS3 media-queries
\item Compatibility with older browsers
\item Progressive enhancement principle
\end{itemize}

And about the interaction: 
\begin{itemize}
\item Device API
\item audio, video, file API
\item Geolocation and canvas
\item Different sensors used
\item Integration of sensors
\end{itemize}
\section{Evaluation}
(Paul)
Requirement: result criteria
Did you understand the interface
Did you understand the process
2 Use Cases: Rhythm clapping (UC1) and Games (UC7)?

\section{Conclusion and Future Work}
(Tom)

\bibliographystyle{plain}
\bibliography{mmm2012}
\end{document}