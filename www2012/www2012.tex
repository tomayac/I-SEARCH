\documentclass{acm_proc_article-sp}

\usepackage[utf8]{inputenc}
\usepackage[T1]{fontenc}

\usepackage[activate=compatibility]{microtype}

% autoref command
\usepackage[pdftex,urlcolor=black,colorlinks=true,linkcolor=black,citecolor=black]{hyperref}
\def\sectionautorefname{Section}
\def\subsectionautorefname{Subsection}

\usepackage{enumitem}

\usepackage{mathtools}

% give emph a normal fontsize
\let\oldemph\emph
\renewcommand{\emph}[1]{\oldemph{\fontsize{9}{9}\selectfont #1}}

% more readable footnote layout
\renewcommand{\footnotesize}{\fontsize{8pt}{10pt}}
\setlength{\footnotesep}{.5cm}

% todo macro
\usepackage{color}
\newcommand{\todo}[1]{\noindent\textcolor{red}{{\bf \{TODO}: #1{\bf \}}}}

% listings and Verbatim environment
\usepackage{fancyvrb}
\usepackage{relsize}
\usepackage{listings}
\usepackage{verbatim}
\newcommand{\defaultlistingsize}{\fontsize{8pt}{9.5pt}}
\newcommand{\inlinelistingsize}{\fontsize{8pt}{11pt}}
\newcommand{\smalllistingsize}{\fontsize{7.5pt}{9.5pt}}
\newcommand{\listingsize}{\defaultlistingsize}
\RecustomVerbatimCommand{\Verb}{Verb}{fontsize=\inlinelistingsize}
\RecustomVerbatimEnvironment{Verbatim}{Verbatim}{fontsize=\defaultlistingsize}
\lstset{frame=lines,captionpos=b,numberbychapter=false,escapechar=§,
        aboveskip=0.5em,belowskip=0em,abovecaptionskip=0em,belowcaptionskip=0em,
framexbottommargin=-1em,
        basicstyle=\ttfamily\listingsize\selectfont}

% use Courier from this point onward
\let\oldttdefault\ttdefault
\renewcommand{\ttdefault}{pcr}
\let\oldurl\url
\renewcommand{\url}[1]{\inlinelistingsize\oldurl{#1}}

% linewrap symbol
\definecolor{grey}{RGB}{130,130,130}
\newcommand{\linewrap}{\raisebox{-.6ex}{\textcolor{grey}{$\hookleftarrow$}}}

% more pleasing quote environment
\usepackage{tikz}
\newcommand*{\openquote}{\tikz[remember picture,overlay,xshift=-7pt,yshift=1pt]
     \node (OQ) {\fontfamily{fxl}\fontsize{16}{16}\selectfont``};\kern0pt}
\newcommand*{\closequote}{\tikz[remember picture,overlay,xshift=2pt,yshift=-4.5pt]
     \node (CQ) {\fontfamily{fxl}\fontsize{16}{16}\selectfont''};}
\renewenvironment{quote}%
{\setlength{\parindent}{1cm}\par\openquote}
{\closequote\vspace{-4.5pt}
}

% bullet numbers
\usepackage{tkz-graph}
\usetikzlibrary{matrix,arrows,decorations.pathmorphing,shapes}
\newcommand{\dobulletnumber}[1]{\node[circle,text=white,fill=gray,anchor=west,inner sep=1pt] {\sffamily #1}}
\newcommand{\bulletnumber}[1]{\tikz[baseline=-2.5,overlay]\dobulletnumber{#1};}
\newcommand{\bulletref}[1]{\tikz[baseline=-2.5]\dobulletnumber{\fontsize{8}{8}\selectfont#1};}

\hyphenation{DBpedia RESTdesc}

\newcommand{\owls}{\mbox{OWL-S}}

\begin{document}

\title{I-SEARCH -- A Multimodal Search Engine based on\\*Rich Unified Content Description (RUCoD)}

\numberofauthors{11}
\author{
\alignauthor
Thomas Steiner\\
	\affaddr{Google Germany GmbH}\\
	\affaddr{ABC-Str. 19}\\
	\affaddr{20354 Hamburg, Germany}\\
	\email{tomac@google.com}
\alignauthor
Jonas Etzold\\
	\affaddr{FH Fulda}
\alignauthor
Paul Grimm\\	
	\affaddr{FH Fulda}
\and
Francesco Nucci\\
	\affaddr{ENG}
\alignauthor
Vincenzo Croce\\
	\affaddr{ENG}
\alignauthor
Antonio Camurri\\
	\affaddr{UNIGE}
}

\additionalauthors{Additional authors: 
  Thanassis Mademlis (ITI),
  Alberto Massari (UNIGE),
  Petros Daras (ITI),
  Apostolos Axenopoulos (ITI), and
  Dimitrios Tzovaras (ITI)
}
\maketitle

\begin{abstract}
\todo{Write Abstract}
\end{abstract}

\category{H.3.4}{Information Systems}{Information Storage and Retrieval}[Semantic Web]
\category{H.3.4}{Information Systems}{Information Storage and Retrieval}[World Wide Web]
\category{H.3.5}{Online Information Services}{Web-based services}

\keywords{\todo{Write Keywords}}

\section{Introduction}
Since the beginning of the age of Web search engines in 1990, the search process is associated with a textual input field.
From the first search engine, Archie~\cite{archie}, to state-of-the-art search engines like WolframAlpha~\cite{wolframalpha}, this fundamental input paradigm has not changed.
In a certain sense the search process has been revolutionized on mobile devices through voice input support like Apple's Siri~\cite{siri} for iOS, Google's Voice Actions~\cite{voiceactions} for Android, and, also from Google, through Voice Search~\cite{voicesearch} for desktop computers.
Support for the human voice as an input modality was mainly driven by shortcomings of mobile keyboards.
However, what is still missing is a truly multimodal search engine. 


\section{Project Goals}

\section{Project Results}

\section{System Demonstration}

\section{Related Work}

\section{Future Work}

\section{Conclusion}

\section{Acknowledgments}
This work was partially supported by the European Commission under Grant No. 248296 FP7 \mbox{I-SEARCH} project.

% back to normal size Computer Modern for URLs in bibliography
\let\ttdefault\oldttdefault
\let\url\oldurl

\bibliographystyle{abbrv}
\bibliography{www2012}

\balancecolumns
\end{document}